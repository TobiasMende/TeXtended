%%%%%%%%%%%%%%%%%%%%%%%%%%%%%%%%%%%%%%%%%%%%%%%%%%%%%%%%%%%%%%%%%%%%%
% A template for IEEE paper.                                        %
% See the documentation of IEEEtran for more configuration details. %
%                                                                   %
% In the documentclass configuration, use one of the following:     %
% [12pt,journal,compsoc]                                            %
% [conference]                                                      % 
% [journal]                                                         %
%%%%%%%%%%%%%%%%%%%%%%%%%%%%%%%%%%%%%%%%%%%%%%%%%%%%%%%%%%%%%%%%%%%%%
\documentclass[12pt,journal,compsoc]{IEEEtran}

\begin{document}

% Define the title:
\title{@@The Title@@ \\ @@ (With Multiple Lines) @@}

% Define the authors here.
%
\author{%
% start of author definition v
%
@@Author~1@@,~\IEEEmembership{@@Membership@@,}
@@Author~2@@,~\IEEEmembership{Membership}%
%
% The following environment ist used to add informations
% for the authors in the footnotes.
\IEEEcompsocitemizethanks{%
%
% one item per author
\IEEEcompsocthanksitem%
@@Author~1 has some properties.@@
\protect\\
@@They can be written in multiple lines.@@
\IEEEcompsocthanksitem%
@@Author~1 has some properties too.@@
}
%
% Use \thanks to access the footnote area, use the commands multiple times
% to produce multiple paragraphs.
\thanks{@@One paragraph in the footnote area.@@}
%
% end of author definitions ^
}

% Define the page headers.
\markboth{@@The jornal@@}%
{@@Author and paper title@@}

% Define the abstract and keywords here
\IEEEcompsoctitleabstractindextext{%
%
% the abstract:
\begin{abstract}
@@The abstract.@@
\end{abstract}
%
% keywords:
\begin{IEEEkeywords}
@@Keyword 1@@, @@Keyword 2@@.
\end{IEEEkeywords}%
}

% insert above informations
\maketitle%
\IEEEdisplaynotcompsoctitleabstractindextext%

%%%%%%%%%%%%%%%%%%%%%%%%%%
% The paper starts here. %
% %%%%%%%%%%%%%%%%%%%%%%%%
\section{@@The first section@@}

\IEEEPARstart{S}{art} @@a paragraph with this command, to obtain a drop 
letter.
Therefore you need at least two lines per paragraph.@@

% Add a appendix at the end.
% Every \section will be one appendix.
\appendices%
\section{@@Title of the appendix@@}
@@Appendix 1.@@

% An appendix don't need a title
\section{}
@@Appendix 2.@@


% At the end add some acknowledgments.
% The title differs in some version, therefore the makkro.
% You should not change this.
\ifCLASSOPTIONcompsoc%
  \section*{Acknowledgments}
\else%
  \section*{Acknowledgment}
\fi%

%%%%%%%%%%%%%%%%%%%%%%%%%%%%%
% Your acknowledgment here. %
%%%%%%%%%%%%%%%%%%%%%%%%%%%%%
@@Thanks!@@

% Define the bibliography.
\ifCLASSOPTIONcaptionsoff%
  \newpage
\fi%
\begin{thebibliography}{1}

%%%%%%%%%%%%%%%%%%%%%%%%%%%%
% Add your bib items here. %
%%%%%%%%%%%%%%%%%%%%%%%%%%%%
\bibitem{IEEEhowto:@@sitekey@@}
@@S.~Name and S.~O. Name, \emph{The Title}.@@
\end{thebibliography}

%%%%%%%%%%%%%%%%%%%%%%%%%%%%%%%%%%%%%%%%%%%%%%%%%%%
% Specify the authors biography at the end.       %
% Use \newpage to balance it in multiple columns. %
%%%%%%%%%%%%%%%%%%%%%%%%%%%%%%%%%%%%%%%%%%%%%%%%%%% 
\begin{IEEEbiography}{@@Author 1@@}
@@Some biography with a photo.@@
\end{IEEEbiography}

\begin{IEEEbiographynophoto}{@@Author 2@@}
@@Some biography without a photo.@@
\end{IEEEbiographynophoto}

% You made it!
\end{document}