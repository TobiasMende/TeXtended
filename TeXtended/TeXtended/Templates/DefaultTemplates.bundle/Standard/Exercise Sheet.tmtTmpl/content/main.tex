%%%%%%%%%%%%%%%%%%%%%%%%%%%%%%%%%%%%%%%%%%%%%%%%%%%%%%%%%%%%%%%%%%%%
% An exercise sheet.                                               %
% Version for students (for the solutions).                        %
%                                                                  %
% For configuration possibility see manual.pdf                     % 
% For more informations visit: http://exercisesheet.googlecode.com %
%%%%%%%%%%%%%%%%%%%%%%%%%%%%%%%%%%%%%%%%%%%%%%%%%%%%%%%%%%%%%%%%%%%%
\documentclass[a4paper]{exercisesheet}

% load properties
% Start your awesome tex project here!

%%%%%%%%%%%%%%%%%%%%%%%%%%%%%%%%%%%%%%%%
% Configuration of the exercise sheet. %
%%%%%%%%%%%%%%%%%%%%%%%%%%%%%%%%%%%%%%%%
\sheetconf{%
    lecture   = @@Specimen of a Title of a Lecture@@,
    lecturer  = @@Prof.~Foo~Bar@@,
    semester  = @@Winter~Semester~2011/2012@@,
    author    = @@Your Name@@,
  }

% Document starts here.
\begin{document}

% Inserting informations from above.
\maketitle%
\tableofcontents%
  
% Simply start a sheet.
\sheet[
    topic={@@Specimen of a Title of an Exercise Sheet@@}
  ]

% A sheet consists out of exercises.
\exercise[
    topic={@@Specimen of a Title of an Exercise@@}
  ]

% And an exercise out of subexercises.
\subexercise[
  topic={@@Specimen of a Title of a Subexercise@@}
    ]

@@The solution!@@

% Hopefully you managed to solve all exercises here. :)
\end{document}
