%%%%%%%%%%%%%%%%%%%%%%%%%%%%%%%%%%%%%%%%%%%%%%%%%%%%%%%%%%%%%%%%%%%%
% An exercise sheet.                                               %
% Version for teachers.                                            %
%                                                                  %
% For configuration possibility see manual.pdf                     % 
% For more informations visit: http://exercisesheet.googlecode.com %
%%%%%%%%%%%%%%%%%%%%%%%%%%%%%%%%%%%%%%%%%%%%%%%%%%%%%%%%%%%%%%%%%%%%
\documentclass[a4paper]{exercisesheet}

% load properties
% Start your awesome tex project here!

%%%%%%%%%%%%%%%%%%%%%%%%%%%%%%%%%%%%%%%%
% Configuration of the exercise sheet. %
%%%%%%%%%%%%%%%%%%%%%%%%%%%%%%%%%%%%%%%%
\sheetconf{%
    narrowfactor = .85,
    lecture      = @@Specimen of a Title of a Lecture@@,
    lecturer     = @@Prof.~Foo~Bar@@,
    semester     = @@Winter~Semester~2011/2012@@,
    author       = @@Your name@@,
    teacher      = @@true@@,
    solutions    = @@false@@,
  }


% Document starts here.
\begin{document}

% Inserting informations from above.
\maketitle%
\tableofcontents%
  
% Simply start a sheet.
\sheet[
    topic={@@Specimen of a Title of an Exercise Sheet@@},
    deadline={@@1.~November~2011@@}
  ]

% A sheet consists out of exercises.
\exercise[
    topic={@@Specimen of a Title of an Exercise@@},
    credits=@@4@@,
    label={@@very_hard_exercise@@}
  ]

% And an exercise out of subexercises.
\subexercise[
  topic={@@Specimen of a Title of a Subexercise@@},
  credits=@@1@@
    ]

@@The exercise!@@

%%%%%%%%%%%%%%%%%%%%%%%%%%%%%%%%%%%%%%%%%%%%%%%%%%%%%%%%%%%%
% You can define solutions.                                % 
% They will only appear, if you set the solutions property %
% in the header to true.                                   %
%%%%%%%%%%%%%%%%%%%%%%%%%%%%%%%%%%%%%%%%%%%%%%%%%%%%%%%%%%%%
\begin{solution}[
    of={@@very_hard_exercise@@}
  ]
  @@The solution.@@
\end{solution}

% Hopefully the sheet is not to hard? :)
\end{document}
